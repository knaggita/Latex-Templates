\documentclass[12.5t,openany]{scrreprt}
\usepackage{listings}
\usepackage[none]{hyphenat}
\usepackage[bookmarks=true]{hyperref}
\hypersetup{
    bookmarks=true,    % show bookmarks bar?
    pdftitle={Software Requirement Specification},    % title
    pdfauthor={Document Author},                     % author
    pdfsubject={TeX and LaTeX},                        % subject of the document
    pdfkeywords={TeX, LaTeX, graphics, images}, % list of keywords
    colorlinks=true,       % false: boxed links; true: colored links
    linkcolor=black,       % color of internal links
    citecolor=black,       % color of links to bibliography
    filecolor=black,        % color of file links
    urlcolor=black,        % color of external links
    linktoc=page            % only page is linked
}
\usepackage {enumerate,mdwlist}
\usepackage{graphicx}
\graphicspath{ {images/}}
\usepackage[a4paper,width=150mm,top=25mm,bottom=25mm,bindingoffset=6mm]{geometry}
\usepackage[toc,page]{appendix}
\setcounter{tocdepth}{3}
\setcounter{secnumdepth}{3}
\setcounter{secnumdepth}{3}
\usepackage{fontspec}
\usepackage{lipsum}
\usepackage{ragged2e} 
\tolerance=1
\emergencystretch=\maxdimen
\hyphenpenalty=10000
\hbadness=10000
\sloppy
\usepackage{algorithm}
\usepackage{algorithmic}
\setmainfont[Ligatures=TeX]{Cambria}
 \usepackage{parskip}
\usepackage{float}
\usepackage{booktabs,tabularx}
\usepackage[nottoc, notlof, notlot]{tocbibind}
\usepackage[style=numeric, backend=bibtex, sorting=none, natbib=true]{biblatex}
\usepackage{fixltx2e}

%add the references bib file
\addbibresource{references.bib}

\usepackage{url}
\usepackage{fancyhdr}
\pagestyle{fancy}
\fancyhf{}
\fancyfoot[LE,RO]{\thepage} 
\fancyfoot[RE,LO]{PROJECT NAME: Software Design Document	}
\renewcommand{\headrulewidth}{0pt}% Default \headrulewidth is 0.4pt
\renewcommand{\footrulewidth}{1.5pt}% Default \footrulewidth is 0pt
 
\fancypagestyle{title}
{\fancyhf{}
\renewcommand{\headrulewidth}{0pt}\renewcommand{\footrulewidth}{1.5pt}
\fancyfoot[L]{PROJECT NAME: Software Design Document}}
\begin{document}

\begin{titlepage}

\textbf{{(Team Name)\\
}}
\textbf{{(Project Time)}}\\
\medskip\medskip\medskip
Software Design Document\\
\vspace{10cm}

\textbf{Prepared by }
\begin{table}[h!]
\setlength{\arrayrulewidth}{0.8pt}
\begin{tabular}{|l|l|l|l|}
\hline
\textbf{\#} & \textbf{Names} & \textbf{Registration No.} & \textbf{Signature} \\ 
\hline
1 & &  &  \\
\hline
2 & & & \\
\hline
3 & & & \\
\hline
\end{tabular}
\end{table}
\\
\textbf{Lab Section} : \\
\textbf{Date } :  
\end{titlepage}
\newpage
\pagenumbering{roman}
\tableofcontents \thispagestyle{fancy}

\newpage
\listoffigures \thispagestyle{fancy}

\newpage
\listoftables \thispagestyle{fancy}


\newpage

\pagenumbering{arabic}

\chapter{INTRODUCTION}\thispagestyle{fancy}

		\section{Purpose }
		
Identify the purpose of this SDD  and  its  intended audience. (e.g. "This  software design document describes the architecture and system design of XX. ...."). 

		\section{Scope}
		
Provide a description and scope of the software and explain the goals, objectives and benefits of your project. This will provide the basis for the brief description of your product. 

		\section{Overview}
Provide an overview of this document and its organization. \\
For example\\

The SDD is divided into 7 chapters. The chapters are:
\begin{itemize}
\item Chapter 1 is the Introduction:  It highlights what the document is about.
\item Chapter 2 provides a system overview, which gives a general description of the functionality, context and design of \textbf{PROJECT NAME}.
\item Chapter 3 is the system architecture. This is the heart of the document. It is a high level overview of how responsibilities of the system were partitioned and assigned to subsystems. It identifies each high level subsystem and the responsibilities assigned to it and describes how these subsystems collaborate with each other in order to achieve the desired functionality. 
\item Chapter 4 explains data design, which details how the information domain of \textbf{PROJECT NAME} is transformed into data structures and describes how the major data or system entities are stored, processed and organized.
\item Chapter 5 explains component design. Here we take a closer look at what each component does in a more systematic way.
\item Chapter 6 gives the user interface designs of the system
\item Chapter 7 gives a cross-reference that traces components and data structures to the requirements in \textbf{PROJECT NAME} SRS document. 
\end{itemize}

\section{Significance}


\section{References}\thispagestyle{fancy}
\textit{This section is optional. }\\
List any documents, if any, which were used as sources of information for the test plan. 

\printbibliography[heading=none]

\section {Definitions and Acronyms}

\textit{This section is optional.}\\
Provide definitions of all terms, acronyms, and abbreviations that might exist to properly interpret the SDD. These definitions should be items used in the SDD that are most likely not known to the audience. 

\begin{table}[H]
\caption{Acronyms, abbreviations and their meanings}
\vspace*{0.2cm}
\begin{tabularx}{\linewidth}{l l X}
\toprule
\textbf{Acronyms and Abbreviations} & \textbf{Meanings}\\
\midrule
OS	& Operating System\\
WSN	& Wireless Sensor Network \\
SDD	 & Software Design Document \\

\bottomrule
\end{tabularx}
\end{table}


\textbf{Definitions }\\
\begin{table}[H]
\caption{Important definitions}
\vspace*{0.2cm}
\begin{tabularx}{\linewidth}{l|p{9cm}}
\toprule
\textbf{Cell}  &  A single element of the schedule characterized by a slotOffset and  channelOffset, and reserved for mote A to transmit to mote B (or for mote B to receive from mote A) within a given slot frame.\\
\\
\textbf{cellID} & It is a number that uniquely identifies a cell.\\
\\
\textbf{Wireless Sensor Network} & A WSN is a network formed by a large number of sensor nodes where each node is equipped
with sensors to detect physical phenomena such as light, heat temperature and pressure.\\
\bottomrule
\end{tabularx}
\end{table}
		
	\newpage	
	\chapter{SYSTEM OVERVIEW}\thispagestyle{fancy}
Give a general description of the functionality, context and design of your project. Provide any 
background information if necessary. 

\newpage
		\chapter{SYSTEM ARCHITECTURE }\thispagestyle{fancy}
		
		\section{Architectural  Design}
Develop a modular program structure and explain the relationships between the modules to achieve the complete  functionality of the system. This  is a  high  level overview of  how responsibilities of the system were partitioned and then assigned to subsystems. Identify each 
high level subsystem and the roles or responsibilities assigned to it. Describe how these subsystems collaborate with each other in order to achieve the desired functionality. Don't go into too much detail about the individual subsystems. The main purpose is to gain a general understanding of how and why the system was decomposed, and how the individual parts work together.  Provide a diagram showing the major subsystems and data repositories and their interconnections. Describe the diagram if required. 

	\section{Decomposition Description}
Provide a decomposition of the subsystems in the architectural design.  Supplement with text as needed. You may choose to give a functional description or an object-�oriented description. For  a  functional  description,  put  top-�level  data  flow  diagram  (DFD)  and structural 
decomposition diagrams.  For an OO description, put subsystem  model, object diagrams, generalization  hierarchy   diagram(s)  (if  any),  aggregation  hierarchy  diagram(s)  (if  any), interface specifications, and sequence diagrams here. 
 
  \section{Design Rationale}
Discuss the rationale for selecting the architecture described in 3.1 including critical issues and  trade/offs  that  were  considered.  You  may  discuss  other  architectures  that  were considered, provided that you explain why you didn't choose them. 

\newpage
		\chapter{DATA DESIGN}\thispagestyle{fancy}
		
	\section{Data Description}
Explain how the information domain of your system is transformed into data structures. Describe how the major data or system entities are stored, processed and organized. List any databases or data storage items. 		

	\section{Data Dictionary}
Alphabetically list the system entities or major data along with their types and descriptions. If you  provided  a  functional  description  in  Section  3.2,  list  all  the  functions  and  function parameters. If you provided an OO description, list the objects and its attributes, methods and 
method parameters. 

\newpage	
		\chapter{COMPONENT DESIGN} \thispagestyle{fancy}
In this section, we take a closer look at what each component does in a more systematic way. If you gave a functional description in section 3.2, provide a summary of your algorithm for each function listed in 3.2 in procedural description language (PDL) or pseudocode. If you gave an 
OO description, summarize each object member function for all the objects listed in 3.2 in PDL or pseudocode.  Describe any local data when necessary. 

\[Q(s_{t}, a_{t})  +=  \alpha[r_{t}  +   \Upsilon.max_{a}Q(s_{t+1}, a_{t+1}) - Q(s_{t}, a_{t})],\]

	\chapter{HUMAN INTERFACE DESIGN} \thispagestyle{fancy}

\section{Overview of the User Interface}
Describe the functionality of the system from the user's perspective. Explain how the user will  be  able to use  your system to complete all the  expected  features and the  feedback information that will be displayed for the user. 

\section{Screen Images}

Display screenshots showing the interface from the user's perspective. These can be hand� drawn or you can use an automated drawing tool.  Just make them as accurate as possible. (Graph paper works well.) 

\section{Screen Objects and Actions}
A discussion of screen objects and actions associated with those objects.

\newpage
\chapter{REQUIREMENTS MATRIX} \thispagestyle{fancy}

Provide a cross�reference that traces components and data structures to the requirements in your SRS document. \\
Use  a  tabular  format  to  show  which  system  components  satisfy  each  of  the  functional requirements from the SRS.  Refer to the functional requirements by the numbers/codes that you gave them in the SRS. 
\begin{table}[h!]
\centering
\setlength{\arrayrulewidth}{0.8pt}
\caption{System components and data structures that satisfy  the functional requirements from the SRS}
\begin{tabular}{|l|p{4.5cm}|l|}
\hline
\textbf{SRS section} &	\textbf{Requirement}	& \textbf{Component and Data structures}\\
\hline
	& & \\
\hline
   & & \\
\hline
  & & \\
\hline
\end{tabular}
\end{table}

\chapter{APPENDICES} \thispagestyle{fancy}
\textit{This section is optional.}\\
Appendices may be included, either directly or by reference, to provide supporting details that could aid in the understanding of the Software Design Document.

\end{document}
